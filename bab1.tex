%-----------------------------------------------------------------------------%
\chapter{\babSatu}
%-----------------------------------------------------------------------------%
%-----------------------------------------------------------------------------%
\section{Latar Belakang}
%-----------------------------------------------------------------------------%
\todo{Ceritakan latar belakang penelitian ini.}


%-----------------------------------------------------------------------------%
\section{Permasalahan}
%-----------------------------------------------------------------------------%
\todo{Jabarkan rumusan masalah yang dibahas di penelitian ini.}

Berdasarkan latar belakang tersebut, maka rumusan masalah dari tugas akhir ini adalah sebagai berikut:
\begin{enumerate}
\item Bagaimana cara kerja sistem untuk xxx yyy zzz yang terganggu tersebut?
\item Bagaimana cara kerja sistem untuk iii jjj kkk dapat maksimal?\\
\textit{Qwerty} yang dimaksud adalah metode untuk aaa bbb ccc.
\item Bagaimana cara menerapkan mmm nnn ooo?
\item Apakah sistem yang dibuat asd fgh ijk?\\
\textit{Asd fgh ijk} dalam hal ini mempunyai arti zxc vbn mkl.
\end{enumerate}

%-----------------------------------------------------------------------------%
\section{Batasan Permasalahan}
%-----------------------------------------------------------------------------%
\todo{Sebutkan batasan-batasan permasalahan penelitian.}

%-----------------------------------------------------------------------------%
\section{Metode Penelitian}
%-----------------------------------------------------------------------------%
\todo{Tuliskan metodologi penelitian yang digunakan.}

%-----------------------------------------------------------------------------%
\section{Sistematika Penulisan}
%-----------------------------------------------------------------------------%
\todo{Jabarkan sistematika penulisan laporan laporan ini. Berikut merupakan contoh sistematika penulisan.}

Sistematika penulisan laporan adalah sebagai berikut:
\begin{itemize}
	\item Bab 1 \babSatu \\
	Bab ini berisi latar belakang, permasalahan, tujuan, metode penelitian, dan sistematika penulisan.
	\item Bab 2 \babDua \\
	Bab ini berisi penjelasan teori, alat, dan perlengkapan yang digunakan.
	\item Bab 3 \babTiga \\
	Bab ini berisi alur kerja dan alur perancangan sistem.	
	\item Bab 4 \babEmpat \\
	Bab ini berisi langkah simulasi dan pengujian yang dilakukan, hasil pengujian, dan analisis dari hasil pengujian yang didapat.
	\item Bab 5 \babLima \\
	Bab ini berisi kesimpulan dan saran tugas akhir ini.
\end{itemize}

