%-----------------------------------------------------------------------------%
\chapter{\babDua}
%-----------------------------------------------------------------------------%
\todo{tambahkan kata-kata pengantar bab 2 disini.

Keterangan di bawah merujuk pada panduan penulisan buku tugas akhir \cite{ta_panduan}}

%-----------------------------------------------------------------------------%
\section{Tubuh Utama Tugas Akhir}
%-----------------------------------------------------------------------------%
Memuat tugas akhir mahasiswa S1. Isi sepenuhnya adalah tanggung jawab mahasiswa S1 dan pembimbingnya. Terdiri dari beberapa bab, diawali bab pendahuluan dan diakhiri dengan daftar pustaka. Jumlah bab tidak standar, disesuaikan dengan keperluan yang wajar untuk mengemukakan tugas akhirnya.

%-----------------------------------------------------------------------------%
\section{Tinjauan Pustaka}
%-----------------------------------------------------------------------------%
Berisi uraian tentang alur pikir dan perkembangan keilmuan topik kajian. Pada hakikatnya, hasil penelitian seorang peneliti bukanlah satu penemuan baru yang berdiri sendiri, melainkan sesuatu yang berkaitan dengan hasil penelitian sebelumnya. Harus dielaborasikan hasil penelitian terdahulu yang berkaitan dengan masalah yang dikaji sedemikian rupa sehingga memberikan gambaran perkembangan pengetahuan yang mendasari penulisan tugas akhir. Mahasiswa ingin menunjukkan bahwa ia menguasai ilmu yang mendasari atau terkait dengan permasalahan yang dikaji.

Tinjauan pustaka hendaknya disusun sesuai dengan urutan perkembangan cabang ilmu pengetahuan yang dikandungnya. Tinjauan pustaka juga berisi ulasan tentang kesimpulan yang terdapat dalam setiap judul dalam daftar pustaka, dan dalam hubungan ini, mahasiswa S1 menunjukkan mengapa dan bagaimana topik kajian serta arah yang akan ditempuhnya dalam menyelesaikan pembahasan topik kajian tersebut. Bila dipandang perlu, tinjauan pustaka dapat disisipkan pada bab-bab isi (sesuai dengan keperluan dan kelaziman pada masing-masing disiplin ilmu) dan tidak harus ditulis dalam bab terpisah.

%-----------------------------------------------------------------------------%
\section{Bab-Bab Utama dalam Tubuh Utama Tugas Akhir}
%-----------------------------------------------------------------------------%
Jumlah bab disesuaikan dengan keperluan. Dalam bab-bab tersebut diuraikan secara rinci cara dan pelaksanaan kerja, hasil pengamatan percobaan atau pengumpulan data dan informasi lapangan, pengolahan data dan informasi, analisis dan pembahasan data serta informasi tersebut, juga pembahasan hasil (\textit{discussion}).

%-----------------------------------------------------------------------------%
\section{Bab Kesimpulan dan Saran}
%-----------------------------------------------------------------------------%
Bab ini memuat elaborasi dan rincian kesimpulan yang dituliskan pada abstrak. Yang dimuat adalah kesimpulan yang diperoleh dari hasil penelitian mahasiswa, bukan kesimpulan dari literatur atau yang merupakan sifat dari sesuatu yang telah umum. Saran untuk kajian lanjutan serta practical implication dari kerja mahasiswa, dapat dituliskan di bab ini.